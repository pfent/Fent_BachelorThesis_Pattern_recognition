\chapter{Conclusion}\label{chapter:conclusion}
To conclude this thesis, we can summarize, that our implementation serves as a proof of concept. We reached the goal of identifying users based solely on acceleration data and showed how to efficiently extract individual features from the acceleration data.

\section{Current state}
Even though the tests we conducted had a relatively small sample size, thus the results are not universally applicable. A 80\% detection rate of our system shows, that it can be used for safer and more convenient authentication mechanisms. %todo: guessing

The performance of processing and classification of acceleration data is in a realistic range for practical use. Entering a password takes several seconds, \ie the same duration needed to process and train a classification algorithm with >100 previous measurements. For real usages, the processing and training phase does not need to take place every time, but can be cached and afterwards simply read from storage.

Even without caching and therefore calculating everything each time, our system performs reasonable well on smartphones as well as on a smartwatch. Authentication on the Sony Smartwatch 3 takes aproximately 30 seconds, which is not great, but still in an acceptable time-frame for watch \glspl{app}. Our implementation however is not suitable for smartrings, because with even lower powered processors, they are probably too slow for data processing directly on the ring. Since smartrings are still in a concept phase, there are good chances that future development in low power processor speeds will result in smartrings with comparable processing speed to smartwatches nowadays.

\section{Future prospects}
For future projects, our implementation can be used with small adaptions to the individual use case. Nevertheless, there are still many possible extensions and improvements left for future work. Currently all of the parameters are statically determined and might not perform the same on all device configurations, especially with fluctuating sensor recording rates. An attempt to work with different recording rates would be interpolation of measurements, \eg cubic spline interpolation to get continuous sensor values.

Overall the Feature extraction steps are in a good shape, but especially implementing specific feature extractions for Android Wear should make it more practical. The biggest potential improvement left to evaluate are neuronal networks to classify measurements. Neuronal networks resulted in immense improvements for speech and image recognition in the last years and many neuronal network implementations have been open sourced recently.
